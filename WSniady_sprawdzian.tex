\documentclass[a4paper]{article}

\usepackage[T1]{fontenc}
% Można też użyć UTF-8
\usepackage[utf8]{inputenc}
% Język
\usepackage[polish]{babel}
% \usepackage[english]{babel}
\usepackage{amsmath}
\usepackage{graphicx}
\author{Wojciech Śniady, nr indeksu 322993}
 \title{Sprawdzian z PWI}
\date{\today}

\begin{document}

 \maketitle
 
 \section{Zadanie 1}
 \begin{verbatim}
ssh-keygen
ssh-copy-id -i /home/wojtek_ubu/.ssh/pwi_rsa.pub mpyzik@pwi.ii.uni.wroc.pl
ssh 'mpyzik@pwi.ii.uni.wroc.pl'
\end{verbatim}

 \section{Zadanie 2}
  \begin{verbatim}
ls \-R ~
touch wojciech_sniady
vi wojciech_sniady
less wojciech_sniady
echo "7 14 21 28 35 42 49 56 63 70 77 84 91 98" >> wojciech_sniady
mkdir testy
mv ./wojciech_sniady ./testy/
exit
scp mpyzik@pwi.ii.uni.wroc.pl:/home/mpyzik/wojciech_sniady/testy/wojciech_sniady ./ 
ssh 'mpyzik@pwi.ii.uni.wroc.pl'

\end{verbatim}

 \section{Zadanie 3}
 \begin{verbatim}
cd ~/wojciech_sniady
mkdir dane
cd dane
hexdump /dev/urandom | head > f1
hexdump /dev/urandom | head > f2
hexdump /dev/urandom | head > f3
hexdump /dev/urandom | head > f4
hexdump /dev/urandom | head > f5
touch concat
cat f1 >> concat
cat f2 >> concat
cat f3 >> concat
cat f4 >> concat
cat f5 >> concat 
grep -E "^0.*([0-9a-f][0-9a-f])\1$" concat > output
wc -l concat 
wc -l output


\end{verbatim}

 \includegraphics[width=150mm,scale=0.5]{/home/wojtek_ubu/Pictures/screen1.png}

  \section{Screen z pisania raportu}
  
   \includegraphics[width=150mm,scale=0.5]{/home/wojtek_ubu/Pictures/screen2.png}

 \end{document}
